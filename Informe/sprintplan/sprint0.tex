\newtcolorbox{UserStoryBox}[1]{%
  enhanced,                % habilita opciones avanzadas
  colback=white,           % color de fondo del contenido
  colframe=black,          % color del borde
  boxrule=0.8pt,           % grosor del borde
  arc=3mm,                 % esquinas redondeadas
  left=3mm, right=3mm, top=2mm, bottom=2mm,
  colbacktitle=black,    % color de fondo del título
  coltitle=white,           % color del texto del título
  fonttitle=\bfseries,      % estilo del título
  halign title=center,      % centrado horizontal del título
  title=#1                  % el título a mostrar
}

\newtcolorbox{TechStoryBox}[1]{%
  enhanced,                % habilita opciones avanzadas
  colback=black,           % color de fondo del contenido
  colframe=gray,          % color del borde
  boxrule=0.8pt,           % grosor del borde
  arc=3mm,                 % esquinas redondeadas
  left=3mm, right=3mm, top=2mm, bottom=2mm,
  colbacktitle=white,    % color de fondo del título
  coltitle=black,           % color del texto del título
  fonttitle=\bfseries,      % estilo del título
  halign title=center,      % centrado horizontal del título
  title=#1,                  % el título a mostrar
  coltext=white
}

\newcommand{\TechStory}[8]{%
\begin{TechStoryBox}{#1} % #1 = Nombre
\textbf{ID:} #2\\
\textbf{Prioridad:} #3\\
\textbf{Estimación:} #4\\
\rule{\linewidth}{0.4pt} % línea horizontal

\textbf{Historia:}\\
\quad \textbf{Como:} #5\\
\quad \textbf{Quiero/necesito:} #6\\
\quad \textbf{Para:} #7\\
\rule{\linewidth}{0.4pt}

\textbf{Criterios de validación:}\\
\begin{itemize}
#8
\end{itemize}
\end{TechStoryBox}
}



\newcommand{\UserStory}[8]{%
\begin{UserStoryBox}{#1} % #1 = Nombre
\textbf{ID:} #2\\
\textbf{Prioridad:} #3\\
\textbf{Estimación:} #4\\
\rule{\linewidth}{0.4pt} % línea horizontal

\textbf{Historia:}\\
\quad \textbf{Como:} #5\\
\quad \textbf{Quiero/necesito:} #6\\
\quad \textbf{Para:} #7\\
\rule{\linewidth}{0.4pt}

\textbf{Criterios de validación:}\\
\begin{itemize}
#8
\end{itemize}
\end{UserStoryBox}
}


\subsection{Sprint 0}
Las \textbf{historias} que se implementan en este sprint son las siguientes:
\UserStory
{Loggear en la APP}
{0}
{0}
{4h}
{Usuario}
{Loggearme en la aplicacion}
{Poder usar la app con mis credenciales}
{
\item Cuando no ponga bien la contraseña me debe salir el error.
\item Cuando me logge, debe haber alguna indicación de que lo he hecho correctamente.
}

\TechStory
{Base de datos postgres}
{1}
{0}
{4h}
{Desarrollador}
{Una base de datos postgres lanzable en mi pc}
{Poder desarrollar la aplicación}
{
\item Tiene que accederse por el puerto 5243.
}

\TechStory
{Mockups figma}
{2}
{0}
{8h}
{Desarrollador}
{Unos mockups en figma}
{Poder desarrollar la aplicación}
{
\item tienen que representar el diseño de la aplicación.
\item tienen que poseer las vista de movil y tablet.
\item tiene que contener el tema de colores de la aplicacion.
}

\UserStory
{Obtener mis havens}
{3}
{0}
{8h}
{Usuario}
{obtener mis havens}
{poder gestionarlos y visualizarlos}
{
\item Se deven ver los havens propios y en los que participo.
\item Mis havens deven tener un botón que me permitan gestionarlos.
\item Los havens deven poderse filtrarion.
}

\TechStory
{Tablero trello}
{4}
{0}
{8h}
{Desarrollador}
{Un tablero trello}
{realizar el desarrollo de los sprints}
{
\item Debe tener todas las historias del sprint 0 a desarrollar.
}

\TechStory
{Plan de sprint}
{5}
{0}
{8h}
{Desarrollador}
{Un plan de sprint}
{realizar el desarrollo de los sprints}
{
\item Debe representar las tareas a realizar.
}

\TechStory
{Backend desplegable en docker}
{6}
{0}
{2h}
{Desarrollador}
{Un backend desplegable en docker}
{aumentar la portabilidad y la facilidad de despliegue del backend}
{
\item Debe representar las tareas a realizar.
}

\UserStory
{Crear Haven}
{7}
{0}
{8h}
{Usuario}
{Crear un haven}
{Tener mi feed personal}
{
\item Si ya hay un haven en el mismo sitio aproximado no debe de poderse crear.
\item Al crearse debe mostrarse un mensaje de creado y navegar al feed del haven.
\item No debe poderme permitir poner el mismo nombre a dos havens.
\item Debo poder poner información de nombre y descripción al haven.
\item En la versión gratuita no puedo crear más de dos havens.
\item Se me debe crear una feed vacía automáticamente al crear el haven.
\item Debe poder ser publico o privado
}

\UserStory
{Eliminar un haven}
{8}
{0}
{4h}
{Usuario}
{Eliminar un haven}
{Poder gestionar mis havens}
{
\item Solo debo poderlo eliminar si es mío.
\item Al eliminarlo debe salirme un mensaje de confirmación.
\item Se me deven dejar una segundos en los que pueda deshacer la acción.
}

\UserStory
{Editar un haven}
{9}
{0}
{4h}
{Usuario}
{Editar un haven}
{Cambiar su nombre o descripción}
{
\item Solo debo poderlo editarlo si es mío.
\item No debo de poder cambiarle el nombre a uno que ya exista.
\item Se me deven dejar una segundos en los que pueda deshacer la acción.
}

\UserStory
{Añadir posts a mi Haven Feed}
{10}
{0}
{4h}
{Usuario}
{Añadir posts en el feed de mi haven}
{Que mis amigos lo vean cuando estén en mi haven}
{
\item Deben poder ser de tipo texto.
\item Deben de poder ser de tipo media (Música, Imagen o Vídeo).
\item El texto debe ser enriquecido (Negrita etc).
}

\UserStory
{Visualizar Haven Feeds}
{11}
{0}
{8h}
{Usuario}
{acceder a una haven feed}
{ver sus publicaciones}
{
\item Solo debo de poder verla si estoy físicamente en el haven.
\item Si no estoy añadido al haven, no devo de poder verlo.
\item Los posts se mostraran de más nuevos a más viejos.
}

\UserStory
{Añadir usuarios a mi haven feed}
{12}
{0}
{8h}
{Usuario}
{poder añadir a usuarios a mi haven feed}
{que vean mis publicaciones}
{
\item Devo de poder enviar invitaciones a amigos.
\item Si pasa alguien por mi haven, me debe salir una notificación que me permita invitarle.
}

\UserStory
{Eliminar posts de mi Haven Feed}
{13}
{0}
{2h}
{Usuario}
{poder eliminar posts de mi haven feed}
{desechar posts que ya no me representan}
{
\item Debo de mostrarme un mensaje de confirmación antes de eliminar.
\item Debo de tener un tiempo en el que pueda deshacerlo.
}

\UserStory
{Registrarme en My Safe Haven}
{14}
{0}
{4h}
{Usuario}
{registrarme en my safe haven}
{poder disfrutar de la red social completa}
{
\item Debe llegarme un correo de confirmación al mail.
\item La contraseña debe de ser segura, y si no lo es mostrar el error.
\item El correo electrónico debe de ser válido.
}

\UserStory
{Navegar en la app}
{15}
{0}
{4h}
{Usuario}
{navegar por la aplicación}
{Poder usar todas sus funciones}
{
\item Se deben poder acceder a las páginas principales.
\item La navegación deve ser fluida y seguir los principios de M3 Design
}




