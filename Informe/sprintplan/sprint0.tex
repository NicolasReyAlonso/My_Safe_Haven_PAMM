\newtcolorbox{UserStoryBox}[1]{%
  enhanced,                % habilita opciones avanzadas
  colback=white,           % color de fondo del contenido
  colframe=black,          % color del borde
  boxrule=0.8pt,           % grosor del borde
  arc=3mm,                 % esquinas redondeadas
  left=3mm, right=3mm, top=2mm, bottom=2mm,
  colbacktitle=black,    % color de fondo del título
  coltitle=white,           % color del texto del título
  fonttitle=\bfseries,      % estilo del título
  halign title=center,      % centrado horizontal del título
  title=#1                  % el título a mostrar
}


\newcommand{\UserStory}[8]{%
\begin{UserStoryBox}{#1} % #1 = Nombre
\textbf{ID:} #2\\
\textbf{Prioridad:} #3\\
\textbf{Estimación:} #4\\
\rule{\linewidth}{0.4pt} % línea horizontal

\textbf{Historia:}\\
\quad \textbf{Como:} #5\\
\quad \textbf{Quiero/necesito:} #6\\
\quad \textbf{Para:} #7\\
\rule{\linewidth}{0.4pt}

\textbf{Criterios de validación:}\\
\begin{itemize}
#8
\end{itemize}
\end{UserStoryBox}
}


\subsection{Sprint 0}
Las \textbf{historias} que se implementan en este sprint son las siguientes:
\UserStory
{Loggear en la APP}
{0}
{0}
{4h}
{Usuario}
{Loggearme en la aplicacion}
{Poder usar la app con mis credenciales}
{
\item Cuando no ponga bien la contraseña me debe salir el error.
\item Cuando me logge, debe haber alguna indicación de que lo he hecho correctamente.
}

