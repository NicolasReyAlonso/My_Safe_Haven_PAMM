\section{Descripción del proyecto}

Este proyecto consiste en el desarrollo de una red social integrada en una aplicación móvil nativa 
denominada \textit{My Safe Haven}. Tomamos como referencia diversas plataformas ampliamente reconocidas, 
como Instagram, Twitter y Tumblr, adoptando algunos de sus conceptos fundamentales, pero introduciendo un enfoque 
innovador que diferencia a nuestra aplicación del resto.

La característica central de \textit{My Safe Haven} radica en que el acceso a las publicaciones 
no es completamente abierto: para visualizar los distintos posts, el usuario debe encontrarse físicamente 
dentro del ``haven'' creado por el autor del contenido. Definimos un \textit{haven} como un tipo particular 
de publicación asociada a un espacio delimitado en el mundo real y compuesta por un nombre identificativo, 
una descripción, un conjunto de elementos que conforman su propio \textit{feed}, así como coordenadas geográficas 
y un radio de alcance que determina su área de influencia.

La creación de un haven también requiere presencia física: el usuario debe encontrarse en la ubicación 
exacta donde desea establecerlo para poder registrarlo en la aplicación. Esta decisión de diseño se inspira en fenómenos 
virales que combinan elementos digitales con interacción física, como el caso de \textit{Pokémon Go}. 
Nuestro objetivo es recuperar esa dimensión híbrida entre lo virtual y lo real, incorporándola en un contexto de red 
social que fomente la exploración, la conexión local y la interacción significativa con el entorno.

De esta manera, \textit{My Safe Haven} busca ofrecer una experiencia social alternativa, basada en la 
proximidad y en la creación de espacios digitales anclados al mundo físico, promoviendo dinámicas más auténticas 
y situadas entre usuarios.
