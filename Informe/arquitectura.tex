\section{Arquitectura}
Las arquitecturas que hemos decidido utilizar para el desarrollo de esta aplicación son las siguientes:
\begin{center}
    \textbf{\textit{Clean Architecture + MVVM + Jetpack (Compose/Views)}}
\end{center}

\subsection{MVVM}
El patrón \textit{Model–View–ViewModel} (MVVM) nos permite separar de forma efectiva la lógica de presentación de la lógica de la interfaz. Esto facilita el mantenimiento del código y reduce el acoplamiento entre componentes. Las principales ventajas por las que se eligió este patrón son:

\begin{itemize}
    \item Permite que las vistas contengan únicamente lógica de interfaz, manteniendo un código más limpio y fácil de extender.
    \item Los \textit{ViewModels} soportan la persistencia del estado incluso frente a cambios de configuración, como la rotación de pantalla.
    \item Facilita el uso de herramientas modernas como \textit{LiveData}, \textit{StateFlow} y \textit{Coroutines}.
    \item Incrementa la capacidad de realizar pruebas unitarias al separar las responsabilidades dentro del flujo de datos.
\end{itemize}

\subsection{Clean Architecture}
La \textit{Clean Architecture} garantiza una separación clara entre las reglas de negocio, los datos y la interfaz de usuario. Esta división en capas mejora considerablemente la escalabilidad y mantenibilidad del sistema. Se seleccionó debido a sus siguientes beneficios:

\begin{itemize}
    \item Define límites bien estructurados mediante capas independientes: dominio, datos y presentación.
    \item Permite encapsular la lógica central de negocio en casos de uso (\textit{use cases}), facilitando la evolución del proyecto sin afectar otras capas.
    \item Facilita la integración de servicios externos, como APIs, sensores de ubicación o bases de datos locales.
    \item Reduce el acoplamiento, lo que permite sustituir componentes (por ejemplo, cambiar la base de datos local) sin modificar la lógica interna.
    \item Mejora considerablemente la capacidad de realizar pruebas, especialmente sobre reglas de negocio.
\end{itemize}

\subsection{Jetpack (Compose/Views)}
El ecosistema \textit{Jetpack} nos proporciona herramientas modernas y altamente optimizadas para el desarrollo de interfaces y funcionalidades móviles. Dentro del proyecto empleamos tanto \textit{Compose} como vistas tradicionales (\textit{Views}), según las necesidades de cada pantalla. Las razones principales para utilizar estas tecnologías son:

\begin{itemize}
    \item \textit{Jetpack Compose} permite crear interfaces declarativas, más intuitivas y fáciles de mantener.
    \item Ofrece integración natural con \textit{ViewModels}, \textit{StateFlow} y otros componentes de arquitectura.
    \item Simplifica el manejo del estado y el renderizado dinámico de la UI.
    \item Framework optimizado por Google para el desarrollo moderno de Android, asegurando compatibilidad y evolución a largo plazo.
    \item La coexistencia con \textit{Views} permite integrar partes previas del sistema o componentes aún no migrados a Compose.
\end{itemize}
